\documentclass{article}

\usepackage{multicol, tikz, enumerate, url}
\usepackage[margin=0.75in]{geometry}

\textwidth=7.0in
\textheight=9.5in


\begin{document}
\pagestyle{empty}

\begin{flushright}
\Huge
\textbf{J. Travis Johnston}

\Large
Research Scientist in Artificial Intelligence \\ for High Performance Computing
\end{flushright}

%\vspace{.15 in}



\LARGE
\noindent \textbf{CONTACT}\normalsize

\noindent \rule{\textwidth}{1px}

	Phone: (865) 241-3512	(office)

	Phone: (308) 530-8401	(cell)		
	
	Email: j.travis.johnston@gmail.com	
	
	Website: https://jtjohnston.github.io/index.html

\vspace{.25 in}



\LARGE
\noindent \textbf{EDUCATION}\normalsize

\noindent \rule{\textwidth}{1px}

	\begin{itemize}
		\item \textbf{University of South Carolina}, Columbia SC

		Ph.D. Mathematics, May 2014.

		Advisor: Linyuan (Lincoln) Lu

		\item \textbf{University of Nebraska-Lincoln}, Lincoln NE

		B.S. Mathematics, May 2009.

		Concurrent Minors in Physics and French

		University Honors Program, Received degree with Highest Distinction

%		\item \textbf{North Platte High School}, North Platte, NE
%
%		Valedictorian, May 2002.
	\end{itemize}

\vspace{.25 in}


%\LARGE
%\noindent \textbf{AREAS OF EXPERTISE}\normalsize
%
%\noindent \rule{\textwidth}{1px}
%
%	\begin{itemize}
%		\item Machine Learning Techniques:
%		
%		Clustering: $k$-means, DBSCAN, hierarchical, etc
%		
%		Modeling: Multi-Linear Regression, $k$-Nearest Neighbors, Polynomial, Multivariate, etc
%
%		Dimensionality Reduction: PCA, Multi-Dimensional Scaling (MDS), Autoencoders
%
%		Support Vector Machines
%
%		Neural Networks (for classification, regression, and dimensionality reduction)
%
%		\item Programming Languages and Frameworks:
%
%		Python (expert proficiency), CUDA C/C++ (moderate proficiency), Maple
%
%		Apache-Spark (expert proficiency), Hadoop (moderate proficiency)
%
%		Caffe (expert proficiency), MXnet (contributor)
%
%	\end{itemize}
%

\LARGE
\noindent \textbf{POSITIONS HELD}\normalsize

\noindent \rule{\textwidth}{1px}

	\begin{itemize}
        \item \textbf{Research Scientist}, Oak Ridge National Lab, June 2018-Present.

        I am a staff scientist in the Computational Data Analytics (CDA) group, Nature Inspired
        Machine Learning (NIML) team.  My research revolves around leveraging supercomputers
        to push the boundaries of what is possible with machine learning and AI.
        I was a Gordon Bell Finalist (2018) for work using the Summit supercomputer to 
        build novel neural network architectures with the ultimate goal of building a 
        self-driving electron microscope capable of building materials atom-by-atom.

		\item \textbf{Postdoctoral Researcher}, Oak Ridge National Lab, July 2016-May 2018.
		
		Advisor: Robert Patton

		Worked on designing Neural Nets (convolutional feed forward nets and fully
        convolutional neural networks) for a variety of tasks including object recognition,
        image segmentation applied to remotely sensed data as well as scientific datasets.
        Designed and implemented a software tool (RAvENNA) to optimize neural network topologies
        in Apache-Spark and scaled it to all of Titan (using 18,000 GPUs).
		The 18,000 node Spark cluster I created is the largest in the world 
		(2nd largest instance Tencent in China with 8,000 nodes).  The code based achieved peak performance
		of 20 PetaFLOPs (single precision) and was featured in an ORNL press release.

		\item \textbf{Postdoctoral Researcher}, University of Delaware, September 2014-July 2016.

		Advisor: Michela Taufer

		Designed algorithms for extreme scale, scientific computing on big data sets with applications to numerical reproducibility, protein folding and prediction, and performance modeling.
		Used wide range of statistical machine learning techniques including surrogate modeling, support vector machines, and clustering algorithms.
		Collaborated with and mentored undergraduate and graduate student researchers.


		\item \textbf{Graduate Teaching Assistant}, University of South Carolina, August 2009-May 2014.

		Instructor of several classes including Precalculus, College Algebra, Finite Mathematics, Calculus I/II, and Business Calculus.
		Gave lectures, recitations, and Maple labs, held office hours, review sessions, wrote, proctored, and graded exams and quizzes, designed activities for Maple labs.
	

		\item \textbf{Mathematics Summer Intern}, National Security Agency, May 2008-August 2008.

		Studied graphics editing techniques.  
		Developed algorithms to detect editing techniques and effects, and implemented these algorithms into image processing software.


		\item \textbf{Teaching Assistant}, University of Nebraska-Lincoln, August 2007-May 2009.

		Taught recitations, held office hours, review sessions, proctored exams and quizzes, graded exams, quizzes and homework.
		

	\end{itemize}

\vspace{.25 in}


\LARGE \textbf{HONORS AND AWARDS}\normalsize

\noindent \rule{\textwidth}{1px}
	
	\begin{itemize}
        \item \textbf{Gordon Bell Finalist (2018)},
        The Gordon Bell Prize is awarded each year (by the ACM) to recognize outstanding
        achievement in high-performance computing. The purpose of the award is to track the
        progress over time of parallel computing, with particular emphasis on rewarding 
        innovation in applying high-performance computing to applications in science,
        engineering, and large-scale data analytics.
        
        I was the technical lead on the project for which our team were finalists.
        I led the effort to both scale the computation and accurately benchmark performance.

		\item \textbf{Director's Discretionary Allocation on Titan}, 3.5M Core-hours allocation on Titan to
        study surrogate-based modelling as a technique for hyper-parameter optimization of neural networks.
        As part of this allocation, I built Apache-Spark on Titan and ran it (using Spark+GPUs) on 18k nodes
        to tune hyper-parameters for a cloud detection network.

        \item \textbf{Best Solution}, Smoky Mountains Conference on Computational Science and Engineering
		Data Challenge (2017).  \url{https://smc-datachallenge.ornl.gov/}

		\item \textbf{2014 Breakthrough Graduate Scholar}

		``The Breakthrough Graduate Scholars program honors USC graduate students who demonstrate excellence in the classroom, and make considerable
		contributions to research and scholarly activities in their field.''

		\item \textbf{Dean's Dissertation Fellowship}

		\$25,000 award from the College of Arts and Sciences.
		Designed to give graduating students more opportunity to travel and conduct research.

		\item \textbf{SIAM Service Award}

		An award from SIAM for ``outstanding efforts and accomplishments on behalf of the SIAM chapter.''

		\item \textbf{Graduate Recruitment Fellowship}
        
	    University of South Carolina Graduate School fellowship, \$8000/yr for 3 years.

		\item \textbf{3rd Place 2009 Cryptanalytic Literature Competition}

		Sponsored by the Kryptos Society; paper written while a summer intern at the National Security Agency (NSA).
		This was the first time student work was ever considered in the competition.

		\item \textbf{Chair's Prize for Outstanding Undergraduate in Mathematics}

		University of Nebraska-Lincoln, Department of Mathematics prize awarded to one graduating math major.

		\item \textbf{Regent's Scholar}
        
		University of Nebraska-Lincoln Board of Regent's Scholarship, full-tuition for 4 years.

	\end{itemize}

\vspace{.25 in}



\noindent \LARGE \textbf{PAPERS}\normalsize

\noindent \rule{\textwidth}{1px}

\noindent (*) indicates that the co-author was a \textbf{graduate} student which I mentored.

\noindent (+) indicates that the co-author was an \textbf{undergraduate} student which I mentored.

\noindent \rule{\textwidth}{1px}

\medskip

\noindent \Large \textbf{Journals} \normalsize

\medskip

	\begin{enumerate}[1.]
        \item R. K. Archibald, M. Doucet, \textbf{Travis Johnston}, S. R. Young, E. Yang, and W. T. Heller.
        Classifying and Analyzing Small-Angle Scattering Data using Weighted $k$-Nearest Neighbors Machine Learning Techniques,
        \textit{Journal of Applied Crystallography}, volume 53, issue 2 (2020).


        \item M. Taufer, T. Estrada, and \textbf{Travis Johnston}.
        A Survey of Algorithms for Transforming Molecular Dynamics Data into Metadata for \textit{in situ} Analytics Based on Machine Learning Methods,
        \textit{Philosophical Transactions of the Royal Society A}, (2020).
        

        \item R. Searles, S. Herbein, \textbf{Travis Johnston}, M. Taufer, and S. Chandrasekaran.
        Creating a Portable, High-Level Graph Analytics Paradigm for Compute and Data-Intensive Applications,
        \textit{International Journal of High Performance Computing and Networking} (IJHPCN), vol 13, (2019).

        \item D. E. Womble, M. Shankar, W. Joubert, \textbf{Travis Johnston}, J. C. Wells, J. A. Nichols.
        Early Experiences on Summit: Data Analytics and AI Applications,
        \textit{IBM Journal of Research and Development}, volume 63, issue 6, (2019).


		\item R. Searles(*), S. Herbein(*), \textbf{Travis Johnston}, M. Taufer, and S. Chandrasekaran.
		A Portable, High-Level Graph Analytics Framework Targeting Distributed, Heterogeneous Systems,
		\textit{International Journal of High Performance Computing and Networking} (IJHPCN), vol 10, (2017).

		
		\item S.M. Cioab\u{a}, W.H. Haemers, M. McGinnis(*), and \textbf{Travis Johnston}.
		Cospectral Mates for the Union of some Classes in the Johnson Association Scheme, \textit{Linear Algebra and its Applications} (LAA),
        volume 539, 219-228 (2018). 

		
		\item \textbf{Travis Johnston} and L. Lu.  Strong Jumps and Lagrangians of Non-Uniform Hypergraphs.  (Submitted, Journal of Combinatorial Theory)
		
		Available on arXiv at: https://arxiv.org/abs/1403.1220


		\item \textbf{Travis Johnston}, B. Zhang(*), A. Liwo, S. Crivelli, and M. Taufer.  In-Situ Data Analytics and Indexing of Protein Trajectories, 
		\textit{J. Comput. Chem.} 2017.


		\item \'{E}. Czabarka, A. Dutle, \textbf{Travis Johnston}, and L. Sz\'{e}kely. Abelian groups yield many large families for the diamond problem,
		\textit{European Journal of Mathematics} 1 (2), 320-328 (2015).
		

		\item \textbf{Travis Johnston}, L. Lu, and K. Milans.  Boolean Algebras and the Lubell Function,
		\textit{Journal of Combinatorial Theory Series A (JCTA)} 136, 174-183 (2015).
		

		\item \textbf{Travis Johnston} and L. Lu. Tur\'an Problems on Non-uniform Hypergraphs,
		\textit{Elec. Journal of Combinatorics}, volume 21, issue 4, (2014).


		\item (See Honor's and Awards) Internally published paper at National Security Agency (while working as summer intern).
		Won award for the paper--this was the first time a student paper was ever considered in the competition.

		
		\item S. M. Nakhmanson, R. Korlacki, \textbf{Travis Johnston}, S. Ducharme, Z. Ge. Vibrational properties of ferroelectric $\beta$-vinylidene fluoride polymers and oligomers,
		\textit{Physical Review B} \textbf{81}, 174120 (2010). 


		\item R.Korlacki, \textbf{Travis Johnston}, et al. Oligo (Vinylidene Fluoride) Langmuir Blodgett films studies by spectroscopic ellipsometry and the density functional theory,
		\textit{Journal of Chemical Physics} (2008).

	\end{enumerate}

\noindent \Large \textbf{Peer Reviewed Conferences and Workshops} \normalsize

\medskip

	\begin{enumerate}[1.]
        \item O. Kotevska, K. Kurte, J. Munk, \textbf{Travis Johnston}, E. McKee(*), K. Perumalla, and H. Zandi.
        RL-HEMS: Reinforcement Learning Based Home Energy Management System for HVAC Energy Optimization,
        \textit{ASHRAE Transactions}, volume 126, (2020).

        \item R. M. Patton, \textbf{Travis Johnston}, S. R. Young, C. D. Schuman, T. E. Potok,
        D. C. Rose, S.-H. Lim, J. Chae, L. Hou, S. Abousamra, D. Samaras, and J. Saltz.  2019.
        Exascale Deep Learning to Accelerate Cancer Research.
        \textit{IEEE International Conference on Big Data (IEEE BigData)},
        Los Angeles, CA, USA, pp 1488-1496.


        \item C. Garcia-Cardona, R. Kannan, \textbf{Travis Johnston}, T. Proffen, K. Page, and S. Seal. 2019.
        Learning to Predict Material Structure from Neutron Scattering Data,
        \textit{IEEE International Conference on Big Data (IEEE BigData)}, Los Angeles, CA, USA, pp 4490-4497.

        \item S. Young, P. Devineni, M. Parsa, \textbf{Travis Johnston}, B. Kay, R. M. Patton, C. D. Schuman, D. C. Rose, and T. E. Potok. 2019.
        Evolving Energy Efficient Convolutional Neural Netowrks,
        \textit{IEEE International Conference on Big Data (IEEE BigData)},
        Los Angeles, CA, USA, pp 4479-4485.

        \item J. Chae, C. D. Schuman, S. R. Young, \textbf{Travis Johnston}, D. C. Rose, R. M. Patton, and T. E. Potok. 2019.
        Visualization System for Evolutionary Neural Networks for Deep Learning,
        \textit{IEEE International Conference on Big Data (IEEE BigData)},
        Los Angeles, CA, USA, pp 4498-4502.

        \item \textbf{Travis Johnston}, S. R. Young, C. D. Schuman, J. Chae, D. D. March, R. M. Patton, and T. E. Potok.  2019.
        Fine-Grained Exploitation of Mixed Precision for Faster CNN Training,
        \textit{2019 IEEE/ACM Workshop on Machine Learning in High Performance Computing Environments (MLHPC)},
        Denver, CO, USA, pp 9-18.

        \item M. Dimovska(*), \textbf{Travis Johnston}, C. D. Schuman, J. P. Mitchell, and T. E. Potok.  2019.
        Multi-Objective Optimization for Size and Resilience of Spiking Neural Networks,
        \textit{2019 IEEE 10th Annual Ubiquitous Computing, Electronics \& Mobile Communication Conference (UEMCON)},
        New York, NY, USA, pp 433-439.


        \item M. Dimovska(*) and \textbf{Travis Johnston}. 2019.
        A Novel Pruning Method for Convolutional Neural Networks Based off Identifying
        Critical Filters.
        \textit{In Proceedings of Practice and Experience in Advanced Research Computing (PEARC19)}.
        ACM, New York, NY, USA, Article 63, 1-7.

        This paper won both the best student paper (ML/AI Track) and the Phil Andrews award for most transformative contribution.


        \item R. M. Patton, \textbf{Travis Johnston}, S. R. Young, C. D. Schuman, D. D. March,
        T. E. Potok, D. C. Rose, S.-H. Lim, T. P. Karnowski, M. A. Ziatdinov, and 
        S. V. Kalinin. 2018.
        167-PFlops deep learning for electron microscopy: 
        from learning physics to atomic manipulation. 
        \textit{In Proceedings of the International Conference for High Performance Computing,
        Networking, Storage, and Analysis (SC '18)}. IEEE Press, Article 50, 11 pages.

        \textbf{Gordon Bell Finalist (2018)}

		\item \textbf{Travis Johnston}, S. Young, D. Hughes, R. Patton, and D. White. 2017.
		Optimizing Convolutional Neural Networks for Cloud Detection.  
        \textit{In Proceedings of the Machine Learning on HPC Environments (MLHPC'17)}. ACM, New York, NY, USA, Article 4, 9 pages.

        \item Steven Young, D. Rose, \textbf{Travis Johnston}, W. Heller, T. Karnowski, T. Potok, R. Patton, G. Perdue, and J. Miller. 2017.
        Evolving Deep Networks Using HPC.
        \textit{In Proceedings of the Machine Learning on HPC Environments (MLHPC'17)}. ACM, New York, NY, USA, Article 7, 7 pages.

	
		\item \textbf{Travis Johnston}, C. Zanin(+), and M. Taufer.  HYPPO: A Hybrid, Piecewise Polynomial Modeling Technique for Non-Smooth Surfaces,
		\textit{In Proceedings of the 28th IEEE Symposium on Computer Architecture and High Performance Computing} (SBAC-PAD), 1-8 (2016).
		
		\textbf{Acceptance Rate: } 27/77 $\sim$ 35\%, one of four best paper finalists

		
		\item M. R. Wyatt II(*), \textbf{Travis Johnston}, M. Papas, and M. Taufer. A Scalable Method for Creating Food Groups Using the NHANES Dataset and MapReduce,
		\textit{ACM Bioinformatics and Computational Biology Conference} (ACM-BCB), 1-10 (2016).

		\textbf{Acceptance rate:} 47/112 $\sim$ 42\%

		
		\item D. Chapp(*), \textbf{Travis Johnston}, and M. Taufer. On the Need Reproducible Numerical Accuracy through Intelligent Runtime Selection of Reduction Algorithms at the Extreme Scale,
		\textit{In Proceedings of the IEEE Cluster Conference}, 166-175 (2015).

		\textbf{Acceptance rate: } 38/157 $\sim$ 24\%


		\item \textbf{Travis Johnston}, M. Alsulmi(*), P. Cicotti, and M. Taufer. Performance Tuning of MapReduce Jobs Using Surrogate-based Modeling, 
		\textit{Procedia Computer Science (International Conference on Computational Science, ICCS)}, volume 51, 49-59 (2015).

		\textbf{Acceptance rate: } 79/304 $\sim$ 26\%

	\end{enumerate}

\medskip

\noindent \textbf{Notes on authorship: }

\noindent For mathematics journals it is typical to list authors in alphabetical order; this convention is followed in the above mathematical papers.
In other disciplines the order of authorship varies.  
Typically the first author is the individual primarily concerned with the collection and analysis of data and has the primary responsibility of writing the paper.
The co-authors are collaborators who contribute to the research and potentially mentor the (first) author.

\medskip

\noindent \textbf{Notes on publication venue: }

\noindent A number of my publications are in proceedings of Computer Science Conferences.
Conferences are the preferred venue for publication and presentation of research results in computer science.
Conference papers are rigorously peer-reviewed comparable to journal papers in other disciplines.
When a paper is selected for publication the work is presented at the conference, assuring a fast research contribution turnaround.
The fast turnaround is essential in a rapidly evolving environment such as Computer Science.

\vspace{.25 in}



%\noindent \LARGE \textbf{INVITED TALKS and POSTERS} \normalsize
%
%\noindent \rule{\textwidth}{1px}
%
%	\begin{enumerate}[1.]
%		\item Oak Ridge National Laboratory, \textbf{Two problems in extremal combinatorics and extreme scale computing}, December 2015.
%		\item Connections in Discrete Mathematics, \textbf{New Non-jump Values for Uniform Hypergraphs}, June 2015.
%		\item International Conference on Computational Science (ICCS) at University of Reykjavik, Iceland, \textbf{Performance Tuning of MapReduce Jobs Using Surrogate-based Modeling}, June 2015.
%		\item $28^{th}$ Cumberland Conference on Combinatorics, Graph Theory, and Computing, \textbf{In-Situ Analysis of Protein Folding Trajectories}, May 2015.
%		\item University of Delaware Discrete Math Seminar, \textbf{Hypergraphs and the Jumping Constant Conjecture}, March 2015.
%		\item University of Delaware Math Club, \textbf{Ramsey Theory: searching large haystacks for small needles (which may not exist)}, February 2015.
%		\item $45^{th}$ Southeastern International Conference on Combinatorics, Graph Theory, and Computing, \textbf{Jumps (and non-jumps) in Hypergraphs}, March 2014.
%		\item Brigham Young University, \textbf{Tur\'an Problems on Hypergraphs,} January 2014.
%		\item University of Chicago Theory of Computing Seminar, \textbf{Connecting Tur\'an Problems on Hypergraphs to Forbidden Subposet Problems}, October 2013.
%		\item USC Discrete Math Seminar, \textbf{Connecting Tur\'an Problems on Hypergraphs to Forbidden Subposet Problems}, September 2013.
%		\item $26^{th}$ Cumberland Conference on Combinatorics, Graph Theory, and Computing, \textbf{Lagrangians and Jumps for Non-uniform Hypergraphs}, May 2013.
%		\item USC Discrete Math Seminar, \textbf{Lagrangians of Non-uniform Hypergraphs}, April 2013.
%		\item AMS Sectional Meeting, Ames IA, presented poster: \textbf{Tur\'an Problems on Non-uniform Hypergraphs}, April 2013.
%		\item SIAM SEAS conference, Knoxville TN, \textbf{Tur\'an Problems on Non-uniform Hypergraphs}, March 2013.
%		\item $44^{th}$ Southeastern International Conference on Combinatorics, Graph Theory, and Computing, \textbf{Tur\'an Problems on Non-uniform Hypergraphs}, March 2013.
%		\item USC Discrete Math Seminar, \textbf{Tur\'an Problems on Non-uniform Hypergraphs}, Fall 2012.
%		\item $25^{th}$ Cumberland Conference on Combinatorics, Graph Theory, and Computing, \textbf{Boolean Algebras, the Lubell function, and more,} May 2012. 
%		\item Graduate Student Conference on Combinatorics, \textbf{Boolean Algebras, the Lubell function, and more,} April 2012.
%		\item Undergraduate Research Conference, presented poster: \textbf{Sorting Signed Permutations Using Cut-And-Paste Operations,} April 2008.
%		\item Midwest Solid State Physics Conference, presented poster: \textbf{Modeling and Spectroscopic Studies of Vinylidene Fluoride Oligomers,} October 2007.
%		\item Undergraduate Research Conference, April 2006 and April 2007, presented posters detailing UCARE research (physics dept).
%
%	\end{enumerate}
%
%\vspace{.25 in}



\noindent \LARGE \textbf{SERVICE} \normalsize

\noindent \rule{\textwidth}{1px}

\medskip

\Large \textbf{Academic/Professional} \normalsize

\medskip


	\begin{itemize}
        \item Served on Program Committee for SC19 (Supercomputing), Machine Learning Track 
		\item Served on Program Committees for IEEE Cluster 2017-2018, Applications Track
		\item Served as ORPA Secretary (Oak Ridge Postgraduate Association) FY 2017.
		\item Served on the Applications track program committee for SBAC-PAD 2016.

		This involved reviewing a half dozen submissions to the conference and assisting with
		the final selection of papers for the Applications track.
	
		\item Reviewed a number of papers for journals including: Order, Discrete Applied Mathematics,
        and the Journal of Parallel and Distributed Computing (JPDC).
%		\item Co-organized the USC Discrete Math Seminar 2013-2014 academic year.
%		\item SIAM (student chapter), President Fall 2012 - Spring 2013.
%		\item SIAM (student chapter), Webmaster Fall 2010 - Spring 2012.
%		\item \textit{King Bee} for Pi Mu Epsilon's annual Integration Bee, March 2013.
%	
%		Created the integrals, the beamer presentation, and hosted the competition.
%		\item \textit{King Bee} for Pi Mu Epsilon's annual Integration Bee, March 2011.
%		\item \textit{Grader} for Pi Mu Epsilon's annual Integration Bee, March 2010.
%		\item Phi Beta Kappa, inducted Spring 2009
%		\item Pi Mu Epsilon, Nebraska Alpha Chapter, Secretary Fall 2007 - Spring 2008.
	\end{itemize}

\medskip

\Large \textbf{Community} \normalsize

\medskip

	\begin{itemize}

		\item Spent 3 days in Louisiana (near Baton Rouge) assisting with flood recovery
        as a part of Mormon Helping Hands, Sept 2016.
		\item Church Organist, Sept 2009 - Sept 2013, Oct 2014-Present.
		\item Assistant Scoutmaster, Sept 2009 - Sept 2013.
		\item Served Full-time mission for The Church of Jesus Christ of Latter Day Saints,
        France Paris Mission (French speaking), Mar 2003 - Mar 2005

	\end{itemize}

\vspace{.25 in}

\noindent \LARGE \textbf{COURSES TAUGHT} \normalsize

\noindent \rule{\textwidth}{1px}

%\noindent \textbf{University of Delaware}
	 \begin{itemize}
	 \item High Performance Computing and Data Analytics (co-taught w/UD faculty), Fall 2015
	 \item High Performance Computing and Data Analytics (co-taught w/UD faculty), Fall 2014

	 Assisted with the preparation of course materials.
	 Gave lectures on machine learning techniques and graph algorithms using MapReduce.
	 \end{itemize}

%\noindent \textbf{University of South Carolina}
%     \begin{itemize}
%	 \item Calculus II (Lecture/Recitations/Labs) Spring 2013
%	 \item Finite Mathematics (Lecture) Fall 2012
%     \item Calculus I (Lecture) Spring 2012
%     \item Intensive College Mathematics (Lecture) Fall 2011
%     \item Precalculus (Lecture) Summer 2011
%     \item Business Calculus (Lecture) Spring 2011
%     \item Precalculus (Lecture) Fall 2010
%     \item Finite Mathematics (Lecture) Summer 2010
%     \item Calculus I (Recitations/Labs) Spring 2010
%     \item Calculus I (Recitations/Labs) Fall 2009
%     \end{itemize}



\end{document}
